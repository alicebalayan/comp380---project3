\documentclass{article}
\title{Project 2 – Developing a Proposal}
\author{
By Lizard Systems Diagnostic\\
Alisiya Balayan\\
Michael Valenzuela\\
Bishoy Abdelmalik\\
Philip Bachman\\
Arian Dehghani\\
Kate Go
}

\begin{document}
\maketitle

\tableofcontents
\pagebreak

\section{Proposal Document}
\subsection{Executive Summary}
\begin{enumerate}
\item Final Cost
  \begin{enumerate}
  \item Final cost of this project is (2-3 sentences)
  \end{enumerate}
\item Brief Timeline of the Project
  \begin{enumerate}
  \item Given that the project will start (2 sentences)
  \end{enumerate}
\item Delivery Date
  \begin{enumerate}
  \item X months from the contract award
  \end{enumerate}
\end{enumerate}

\subsection{High Level Schedule}
\begin{enumerate}
\item Interactions (meetings, trainings, etc.)
  \begin{enumerate}
  \item During Requirements Phase
  \item Release Phase
  \item Post-release Phase
  \end{enumerate}
\item Product Delivery
  \begin{enumerate}
  \item Final Product will be delivered
  \end{enumerate}
\item Why Lizard Systems Diagnostic?
  \begin{enumerate}
  \item Why should the costumer go with our company?
  \end{enumerate}
\end{enumerate}

\subsection{Storyboards}
\begin{itemize}
\item Deliverables - description goes here
\item Tasks
\item Issues
\item Action Items
\item Decisions
\item Resources
\item Risks
\item Requirements
\item Changes
\item Reference Documents
\item Components
\item Defects
\end{itemize}

\subsection{Requirements Traceability Matrix}
Refer to \texttt{costEstimate.xlsx} sheet ``Matrix''
\section{Backup Material}
\subsection{Cost Estimate}
Refer to \texttt{costEstimate.xlsx} sheet ``Cost Estimating''

\subsection{Duration Estimate}
\subsubsection{Assumptions}
\begin{itemize}
\item Schedule
  \begin{itemize}
  \item Assuming employees will work 8-hour shifts, 5 business days a week which in total is a full-time employee (40 hours a week).
  \item Assuming there will be an occasional over time during the implementation phase.
  \end{itemize}
\item Resource Distribution
  \begin{itemize}
  \item Assuming there will be an unlimited resource distribution due to executive support.
  \item Assuming all employees have different skill levels, therefore some may take longer/faster than others to complete their assigned tasks.
  \item Assuming a certain employee is not making less if any progress on the task, he/she will be reassigned a new task that will better match his/her skill level.
  \item Assuming the number of tasks per employee will depend on the seniority level of an employee (i.e., senior level developers might receive more tasks both technical and non-technical opposed to junior level developers).
  \item Assuming one employee will be assigned multiple tasks (i.e., one employee per 3 tasks).
  \end{itemize}
\item Meetings and Trainings
  \begin{itemize}
  \item Assuming it will take 5 business days’ worth of meetings and trainings before starting the project (i.e., coding).
  \item Assuming there will be meetings with the customer during the Requirements phase.
  \item Assuming there will be meetings with the customer after the final product has been delivered. 
  \item Assuming the customer will need training on how to use the software properly.
  \end{itemize}
\item Dependencies
  \begin{itemize}
  \item Assuming tasks will be performed in parallel as much as possible due to unlimited resources.
  \end{itemize}
\item Tools
  \begin{itemize}
  \item Assuming software will be hosted on AWS, therefore will not affect the availability of the product.
  \item Assuming cloud-based backup, system will be used which will not affect the failure of the backup requirement.
  \item Assuming SQL will be used to setup the database.
  \end{itemize}
\item Processes
  \begin{itemize}
  \item Assuming Waterfall duration estimating method was used to estimate the life cycle of the project.
  \item Assuming Task Dependencies Calendar will be used for Detailed Scheduling Method.
  \item Assuming Microsoft UI will be used as an interface design.
  \item Assuming Requirements Traceability Matrix will be used to ensure that all customer requirements have been addressed and matched with our proposal documents.
  \end{itemize}
\item Cost
  \begin{itemize}
  \item Assuming 40\% of overhead and 15\% of profit for the software project. 
  \item Assuming unlimited executive support for the project cost.
  \item Assuming the development and execution of proposed requirements will be covered.
  \end{itemize}
\item Code reusing
  \begin{itemize}
  \item Assuming some chunks of code will be reused to save time and efforts.
  \end{itemize}
\item Software Testing
  \begin{itemize}
  \item Assuming no additional software testers will be hired, or any other testing agencies used to test our software.
  \item Assuming all the software testing will be done by existing employees.
  \end{itemize}
\item Units
  \begin{itemize}
  \item Assuming the business days units of measure are being used.
  \item Assuming business days units of measure are being converted to weeks, months and year(s) for the total summary.
  \end{itemize}
\item Proposed Requirements
  \begin{itemize}
  \item Assuming cost estimate, detailed schedule and requirements traceability matrix contain proposed requirements.
  \item Rest is in Progress.
  \end{itemize}
\end{itemize}

\subsubsection{Duration Estimating Method(s)}
\par In order to estimate the duration of this project, waterfall
software development process model was used. Overall idea of this
model is that tasks occur sequentially one after another where the
output or result from one task will fall into the next task, hence the
name waterfall implies. First, we started with evaluating all the
requirements to ensure that everything is addressed properly by
communicating with the customer. Once the requirements phase is
complete, we moved on to the next phase which is designing the project
and it consists of both high level and detailed designs. The next
stage after designing is implementation where we estimated how long it
will take to code each requirement. Finally, once all the above is
complete, we moved on to estimating the duration of the testing phase
which is the final phase before releasing the product to the customer.

\subsubsection{Advantages}
\begin{itemize}
\item Requirements are understood better because a lot of time is spent on the requirements phase to ensure that all the customer needs are understood correctly by interactions and verifications of the given requirements with the customer. 
\item Easier to manage because a software project may be tracked due to sequential flows from one phase to another. Therefore, you cannot move on to the next phase until the current phase is completed.
\end{itemize}

\subsubsection{Disadvantages}
\begin{itemize}
\item Customer interactions are limited and happen only in the beginning of the process during the requirements phase to ensure that requirements are well understood and at the end of process when the product is being delivered to the customer.
\item Working software is not produced until late during the cycle since waterfall model doesn’t follow early releases methodology. Therefore, there is no user involvement throughout the development cycle and if there are any issues it will be harder to go back and start all over.
\item Waterfall model is not suitable for ongoing and long-term projects because the deeper you are in the process (i.e., implementation phase), the harder it will be to develop the project. For example, most ongoing projects require maintenance which makes you go back in different phases to adjust certain needs of the project, however it is harder to do with waterfall model. 
\end{itemize}

\subsection{Detailed Schedule}
Refer to \texttt{costEstimate.xlsx} sheet ``Schedule''

\end{document}
